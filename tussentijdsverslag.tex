\documentclass{article}

\usepackage[dutch]{babel}
\usepackage{graphicx, titling}
\usepackage{subfiles}
\usepackage{pdfpages}
\usepackage[toc,page]{appendix}


\title{Communicatieplatform}
\author{Thomas Varheust \\
	Ruben Min Jou \\
	Manon Delobelle
                            }
\date{}

\begin{document}
\pagestyle{empty}

\begin{center}
	{{\Large Subfaculteit wetenschappen}
	
	\vspace{1cm}
	
	\includegraphics[width=6cm]{2013-kulak-cmyk-highres.jpg}
	
	\vspace{1cm}
	
	\Large Probleemoplossen en ontwerpen 3}
	
	\vspace{2cm}

	{\Huge \textbf{Communicatieplatform}}
	
	\vspace{1cm}
	
	{\Large \textbf{Let the dogs out}}
	
	\vspace{1cm}
	
	{\Large \textbf{Thomas Varheust}}\\
	{\Large \textbf{Ruben Min Jou}}\\
	{\Large \textbf{Manon Delobelle}}\\

\end{center}

\vspace{2cm}
{\Large Titularis : Koen Van Den Abeele}
\vspace{1cm}

{\Large Begeleider : Charlotte Deconinck}


\vspace{1cm}

\begin{center}
	{\Large Academiejaar 2021 - 2022}
\end{center}


\newpage
\tableofcontents
\newpage
\section*{Inleiding}
\normalsize
Blinde mensen hebben vaak honden nodig om hen te helpen in het dagelijks leven. Die blindengeleidehonden hebben daarvoor eerst een opleiding nodig. Ze moeten zowel  stressbestendig als gehoorzaam zijn in drukke situaties om zo hun baasje optimaal te begeleiden. Hiervoor verblijven ze gedurende een periode bij een pleeggezin en worden ze getraind door hun gezin en trainers. In die periode is goede communicatie erg belangrijk. Zo kunnen de blindengeleidehonden optimaal worden opgeleid om later een blinde te begeleiden. \\
Op dit moment verloopt die communicatie niet altijd vlot met opleidingscentra. Dit heeft ongunstige gevolgen voor de opleiding van de hond. Daarvoor zoeken wij een oplossing. 

Het doel is om een communicatieplatform te creëren voor die pleeggezinnen en hun trainers.
Op die manier kan het gezin communiceren met de trainer en omgekeerd. Hiervoor heeft iedereen een persoonlijke login waarmee ze gegevens kunnen vinden over de opleiding van een hond. Ze kunnen een kalender raadplegen waarop ze de trainingen en toediening van medicatie kunnen bekijken. Daarnaast is het ook mogelijk om een dagboek bij te houden van de verschillende activiteiten die het gezin doen met de hond zoals bijvoorbeeld een terrasje doen. Die data kan worden opgevraagd via een overzicht om inzicht te krijgen in welke activiteiten er meer getraind moeten worden. Het gezin kan ook verslagen van trainingen opvragen van de trainers. \\
Hiervoor is zowel een Android-app als een website nodig: De gezinnen kunnen via hun app herinneringen krijgen voor trainingen van de hond, of voor medicijnen dat de hond moet innemen\texttt{...} Daarnaast kan de trainer via zijn computer de verslagen van trainingen uploaden die belangrijk zijn om te weten waarop er nog getraind moet worden.

\subsection*{Integratie van vakken}
Het hoofddoel is een app en website bouwen die samen een communicatieplatform vormen. Het vak 'Beginselen van programmeren' hebben we hierbij nodig aangezien programmeren een hoofdzaak is bij het bouwen van zo'n app en website. In die opleiding hebben we de taal 'Python' geleerd en inzichten gecreëerd die handig kunnen zijn in dit project.


\newpage
\section{Studie voor tools}
\subsection{Taal}
Er bestaan enorm veel programmeertalen. We kenden enkel Python, dus de andere talen zouden we moeten aanleren. Aangezien Python niet de beste taal bleek te zijn om een app te ontwikkelen, zochten we verder. Doordat we de taal nog moeten leren kennen, hebben we beslist om bekendere talen te gebruiken. Omdat we het op die manier makkelijker konden aanleren aan de hand van vele online video's. Wij hebben op basis van deze voorkennis en onderzoek meteen een selectie gemaakt tot 6 talen: Java, JavaScript, Python, Kotlin, C en C++. Daarna zochten we uit welke talen het beste waren voor onze opdracht. \\

Java en Kotlin blijken de beste programmeertalen te zijn om een app te bouwen. Zowel Java als Kotlin zijn officiële programmeertalen voor $Android development$, daardoor sluiten ze goed aan op het besturingssysteem Android. Kotlin is makkelijk aan te leren, maar bestaat pas sinds 2011 waardoor er niet zoveel documentatie online staat. Online documentatie is voor sommige onderdelen (zoals bij chat) erg belangrijk doordat het erg moeilijk is om daarvoor van nul te starten met een taal die voor ons niet volledig gekend is. Om een webapp\footnote{Een webapp is een app waarvoor het mogelijk is om die te benaderen met een browser.} te ontwikkelen wordt er aangeraden om dit met talen te doen die populair zijn in de webdevelopment wereld, zoals Java. We hebben dus gekozen om Java te gebruiken als programmeertaal voor de app.
%of een hybride app?

Voor de \it{front}-end\footnote{De front-end van een website is deel van het programma dat zichtbaar is voor de gebruiker.} ontwikkeling van de website kozen we ervoor om gebruik te maken van HTML, CSS en Javascript. Dit zijn veruit de meest gebruikte talen voor de ontwikkeling van een \it{front-end} (Meer dan 94\% van de websites maakt gebruik van HTML, CSS en Javascript). Ook zijn HTML en CSS zijn eenvoudige talen om aan te leren en heel overzichtelijk. HTML wordt gebruikt voor de structuur. Voor de presentatie wordt CSS gebruikt en voor het gedrag van de site JavaScript.

\newpage
\subsection{App}
Er bestaan veel tools om een app te ontwikkelen voor het besturingssysteem Android. De tool moest voldoen aan verschillende vereisten. Allereerst moet het compatibel zijn met Android. Het moet ook mogelijk zijn om de data weg te schrijven naar een database zodat er een website aan gekoppeld kan worden. Daarnaast hadden we geen budget waardoor we een gratis platform zochten. Tot slot moet de tool ook aan alle vereisten voor ons probleem voldoen, zo moet het mogelijk zijn om documenten up te loaden, om een dagboek bij te houden en om een chatfunctie te installeren. 

Aangezien wij een Android-app moeten ontwikkelen was de keuze snel gemaakt; we kozen voor Android Studio. Het brengt app-ontwikkeling eenvoudiger dankzij de 'lay-outeditor', de 'APK-analyzer' en 'Vector Asset Studio'.. Die eerste zorgt ervoor dat een lay-out snel gebouwd kan worden en er later makkelijk onderdelen kunnen gewijzigd worden. De 'Analyzer' zorgt dat de APK\footnote{APK is een bestandsformaat voor software van het Android-platform.}-grootte verkleint.
Dankzij de 'Vector Asset Studio' kunnen verschillende bestanden gegenereerd worden voor alle schermdichtheden.
Android Studio is dus specifiek gericht op het Android platform. In Android Studio is het ook makkelijk om programmeertalen Kotlin, Java en C++ te combineren bij het programmeren van de app. Het is ook mogelijk om aan de hand van Android-emulators de app te testen.
Daarnaast kan Android Studio ook gebruikt worden met hulpprogramma's voor een beter en efficiënter groepswerk. Het kan gesynchroniseerd worden met GitHub\footnote{Github is een versiebeheersysteem. Telkens er een wijziging wordt gedaan in de code, wordt deze opgeslaan. Deze kunnen later gepusht worden zodat elk teamlid de wijzigingen ziet.} waardoor alle groepsleden kunnen werken met de meest recente versie van de app. De verschillende functionaliteiten kunnen via deelprogramma's onafhankelijk worden ontwikkeld en getest. 
Tot slot kan Android Studio makkelijk gekoppeld worden met de database Firebase (Dit komt later nog aan bod in hoofdstuk \ref{sec:Database} op pagina \pageref{sec:Database}.) doordat Firebase-services rechtstreeks kunnen geïntegreerd worden vanuit Android Studio.

\newpage
\subsection{Website}
Voor de ontwikkeling van de \it{back-end}\footnote{De back-end van een website is deel van het programma dat onzichtbaar is voor de gebruiker.} van de website bestaan er veel verschillende tools. We zochten een gratis tool die op verschillende \it{browsers} werkt. Hiervoor leken Python Flask, PHP en Node.js ons de beste opties. We hebben ervaring met Python door het vak "Beginselen van programmeren" en moeten Javascript beheersen voor de \it{front-end} van de website en omdat we maar 12 weken hebben voor ons project besloten we om PHP niet te gebruiken omdat we dan nog een nieuwe syntax moeten leren kennen. De \it{event-driven} architectuur van Node.js maakt het eenvoudiger om een chat-app te implementeren, waardoor onze keuze gemaakt was. Node.js is een opensource en multiplatform van JavaScript.

\newpage
\subsection{Database} \label{sec:Database}
% jason: dictonairies in dictionaries --> non sequal 
Android Studio werkt rechtstreeks via stapsgewijze procedures met Firebase. Daarnaast kan de app ook gekoppeld worden met Google Cloud Endpoints en andere modules die ontwikkeld zijn voor Google App Engine.

\newpage
\section{Resultaten}
Op dit moment zijn we eraan uit welke tools we gaan gebruiken: Android Studio voor de app, $iets$ voor de website en Firebase voor de database. 
Daarna hebben we gezocht naar een structuur (die weergegeven wordt op pagina \pageref{flowchart}), die we wouden bereiken voor het communicatieplatform. Hiervoor hebben we verschillende pagina's nodig: login, menu, chatoverzicht, documenten, kalender, dagboek, activiteiten toevoegen, overzicht van activiteiten, documenten toevoegen en chatberichten. \\
Voor de app is er al een uitgewerkt idee voor het design. Hierbij zijn de verschillende pagina's al ontworpen en is het al mogelijk om door te klikken tussen de pagina's. Daarnaast is het gelukt om de app te verbinden met de Firebase. \\
Met diezelfde database is het ook gelukt om de website te verbinden. Voor de website lukt het inlog-systeem al en wordt er gewerkt aan het dagboekje. \\
Firebase is dus gekoppeld met zowel de app als de website. Aan de structuur om alles weg te schrijven vanuit de app en website naar de database wordt er gewerkt. \\

Er moeten nog veel pagina's geprogrammeerd worden bij de app. Ook bij de website is er nog veel werk aangezien er nog moet gekeken worden voor zowel de pagina's te implementeren als het design voor die pagina's. Tot slot moet alles nog getest worden, of alles juist doorgegeven wordt naar de database en er juist in wordt opgeslaan.
\includepdf[pages=-]{Flowchart} \label{flowchart}
%voorlopige resultaten + hoe verder in toekomst

\section{Samenwerking}
\subsection{Verantwoordelijheden}
Bij het begin van het project werden er rollen en dus verantwoordelijkheden gecreëerd. Thomas heeft de meeste voorkennis bij het ontwerpen van een website, dus werd de verantwoordelijke voor de site. Daarnaast heeft Ruben de verantwoordelijkheid van de app op zich genomen. En tot slot is Manon verantwoordelijk voor het design van het volledige platform en voor de rapportering van ons project.
\subsection{Taakverdeling}
Wij hebben samen gezocht naar verschillende tools om zo'n website en app te bouwen en met elkaar te verbinden via een database. Toen we een idee hadden welke tools we voor het project gingen gebruiken, werd de Gantt-grafiek (in bijlage op pagina \pageref{gantt-grafiek}) opgesteld en hebben we de taken (in bijlage op pagina \pageref{takenstructuur}) opgesplitst. \\
Thomas is toen begonnen met het connecteren van de database en de website. En hij heeft erna al pagina's van de website geïmplementeerd en de structuur van de database gemaakt. 
Ruben heeft gekeken voor het connecteren van de database en de app. En hij heeft ervoor gezocht dat Android Studio via Github functioneert om zo een vlotter groepswerk te creëren. 
Manon is begonnen met de app te ontwerpen: het design van de verschillende pagina's en de doorverwijzingen tussen de pagina's. Daarna is ze begonnen met het tussentijds verslag te schrijven, er structuur in te brengen en bijlagen te ontwerpen. \\
Voor het maken van het tussentijds verslag hebben we later ook de taken verdeeld. We namen elk een of meerdere onderdelen voor zich. Iedereen kreeg het onderdeel waarvan hij of zij het meest vanaf wist. Tot slot heeft iedereen het verslag doorlezen en goedgekeurd.


\section*{Conclusie}


\newpage
\begin{appendices}
\includepdf[pages=-]{Ganttchart} \label{gantt-grafiek}
\includepdf[pages=-]{Takenstructuur} \label{takenstructuur}
\end{appendices}

\bibliography{Bibliografie}
\bibliographystyle{unsrt}

\end{document}