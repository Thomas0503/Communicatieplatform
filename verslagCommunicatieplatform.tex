\documentclass{article}
\usepackage{hyperref}
\usepackage[dutch]{babel}
\usepackage{graphicx, titling}
\usepackage{subfiles}
\usepackage{pdfpages}
\usepackage[toc,page]{appendix}
\usepackage{amsmath}
\usepackage{pdflscape}


\title{Communicatieplatform}
\author{Thomas Varheust \\
	Ruben Min Jou \\
	Manon Delobelle
}
\date{}
%\pagenumbering{arabic}

\begin{document}
	\pagestyle{plain}
	
	\begin{center}
		{{\Large Subfaculteit wetenschappen}
			
			\vspace{1cm}
			
			\includegraphics[width=6cm]{2013-kulak-cmyk-highres.jpg}
			
			\vspace{1cm}
			
			\Large Probleemoplossen en ontwerpen 3}
		
		\vspace{2cm}
		
		{\Huge \textbf{Communicatieplatform}}
		
		\vspace{1cm}
		
		{\Large \textbf{Let the dogs out}}
		
		\vspace{1cm}
		
		{\Large \textbf{Thomas Varheust}}\\
		{\Large \textbf{Ruben Min Jou}}\\
		{\Large \textbf{Manon Delobelle}}\\
		
	\end{center}
	
	\vspace{2cm}
	{\Large Titularis : Koen Van Den Abeele}
	\vspace{1cm}
	
	{\Large Begeleider : Charlotte Deconinck}
	
	
	\vspace{1cm}
	
	\begin{center}
		{\Large Academiejaar 2021 - 2022}
	\end{center}
	
	
	\newpage
	\tableofcontents
	\newpage
	\section*{Inleiding}
	\normalsize
	Blinde mensen hebben vaak honden nodig om hen te helpen in het dagelijks leven. Die blindengeleidehonden hebben daarvoor eerst een opleiding nodig. Ze moeten zowel stressbestendig als gehoorzaam zijn in drukke situaties om zo hun baasje optimaal te begeleiden. Hiervoor verblijven ze gedurende een periode bij een pleeggezin en worden ze getraind door hun gezin en trainers. In die periode is goede communicatie erg belangrijk. Zo kunnen de blindengeleidehonden optimaal worden opgeleid om later een blinde te begeleiden. 
	\\
	Op dit moment verloopt die communicatie niet altijd vlot met opleidingscentra. Dit heeft ongunstige gevolgen voor de opleiding van de hond. Daarvoor zoeken wij een oplossing.
	\\\\
	Het doel is om een communicatieplatform te creëren voor die pleeggezinnen en hun trainers.
	Op die manier kan het gezin communiceren met de trainer en omgekeerd. Hiervoor heeft iedereen een persoonlijke login waarmee ze gegevens kunnen vinden over de opleiding van hun hond. Ze kunnen een kalender raadplegen waarop ze de trainingen en toediening van medicatie kunnen bekijken. Daarnaast is het ook mogelijk om een dagboek bij te houden van de verschillende activiteiten die het gezin kan doen met de hond, zoals bijvoorbeeld een terrasje doen. Die data kan worden opgevraagd via een overzicht om inzicht te krijgen in welke activiteiten er meer getraind moet worden. Het gezin kan ook verslagen van trainingen opvragen van de trainers.
	\\\\
	Hiervoor is zowel een Android-app als een website nodig: de gezinnen kunnen via hun app herinneringen krijgen voor trainingen van de hond, voor medicijnen die de hond moet innemen \texttt{...} Daarnaast kan de trainer via zijn computer de verslagen van trainingen uploaden, die belangrijk zijn om te weten waarop er nog getraind moet worden. Trainers en pleeggezinnen kunnen echter ook gebruik maken van het andere platform, indien ze dit willen.
	
	\subsection*{Integratie van vakken}
	Het hoofddoel is een app en een website te bouwen die samen een communicatieplatform vormen. Het vak 'Beginselen van programmeren' hebben we hierbij nodig, aangezien programmeren een hoofdzaak is bij het bouwen van zo'n app en website. In die opleiding hebben we de taal \verb!Python! geleerd en inzichten gecreëerd die handig kunnen zijn in dit project. \\
	In P\&O1 leerden we hoe we professioneel een verslag kunnen opstellen en een presentatie kunnen geven. Die vaardigheden werden nog eens getraind in P\&O2. Deze zijn ook handig in P\&O3 aangezien we voor de rapportering van ons project opnieuw een eindverslag indienen en presentatie met demo zullen geven. \\
	
	
	\newpage
	\section{Studie voor tools}
	\subsection{Programmertalen} 
	\verb!Java! en \verb!Kotlin! blijken de beste programmeertalen te zijn om een app te bouwen volgens bron \cite{Programmeertaal}. Aangezien zowel \verb!Java! als \verb!Kotlin! officiële programmeertalen zijn voor Android development, sluiten ze goed aan op het besturingssysteem Android. \verb!Java! bestaat al langer dan \verb!Kotlin!. Hierdoor bestaat er meer online documentatie voor \verb!Java!, die nuttig is voor onderdelen die moeilijk te implementeren zijn (zoals bij chat). Daarom hebben we gekozen om \verb!Java! te gebruiken als programmeertaal voor de app.
	\\\\%of een hybride app?
	Voor de \textit{front-end}\footnote{De \textit{front-end} van een website is deel van het programma dat zichtbaar is voor de gebruiker. Dit zorgt voor het visuele gedeelte van de site.} ontwikkeling van de website kozen we ervoor om gebruik te maken van \verb!HTML!, \verb!CSS! en \verb!JavaScript!. Dit zijn veruit de meest gebruikte talen  \cite{JavaScript_for_web-app} voor de ontwikkeling van een \textit{front-end}.
	
	\subsection{App}
	Er bestaan veel tools om een app te ontwikkelen voor het besturingssysteem Android. De tool die wij wilden gebruiken, moest voldoen aan verschillende vereisten. Het moet compatibel zijn met Android, het moet mogelijk zijn om de data weg te schrijven naar een database zodat er een website aan gekoppeld kan worden,  en we moeten alle klantenvereisten er mee kunnen implementeren. Zo moet het mogelijk zijn om documenten te uploaden, om een dagboek bij te houden en om een chatfunctie te installeren. 
	\\\\
	We kozen ervoor om Android Studio \cite{AndroidStudio} te gebruiken
	Deze IDE, die mede ontwikkeld is door Google, is specifiek gericht op het Android platform en er kan makkelijk op getest worden aan de hand van Android-emulators. Een ander voordeel is de ingebouwde \textit{lay-outeditor}, waarmee relatief gemakkelijk de lay-out van de app ontwikkeld kan worden.
	Daarnaast kan Android Studio gesynchroniseerd worden met GitHub\footnote{GitHub is een versiebeheersysteem. Telkens er een wijziging wordt gedaan in de code, wordt deze opgeslagen. Deze kunnen later gepusht worden zodat elk teamlid de wijzigingen ziet.}. Dit vereenvoudigt de samenwerking tussen verschillende groepsleden. Daarnaast kunnen de verschillende functionaliteiten via deelprogramma's zo ook onafhankelijk worden ontwikkeld en getest.
	\\
	Tot slot kan Android Studio makkelijk gekoppeld worden met de databases van Firebase (dit komt aan bod in hoofdstuk \ref{sec:Database}) doordat de Firebase-services rechtstreeks met Android Studio kunnen gekoppeld worden.
	
	\subsection{Website}
	Voor de ontwikkeling van de \textit{back-end}\footnote{De \textit{back-end} van een website is deel van het programma dat onzichtbaar is voor de gebruiker. Dit zorgt ervoor dat de \textit{front-end} van de website werkt.} van de website bestaan er veel verschillende tools. \verb!Node.js! leek ons de beste optie. De \textit{event-driven} architectuur van \verb!Node.js! maakt het eenvoudiger om een chat-app te implementeren en is snel en makkelijk te begrijpen, waardoor onze keuze gemaakt was. \verb!Node.js! is een open source en multiplatform van de \verb!JavaScript-runtime-omgeving!.
	
	\subsection{Database} \label{sec:Database}
	% jason: dictonairies in dictionaries --> non sequal 
	Zowel de trainers als de pleeggezinnen zullen bij het gebruik van de app en de website informatie, zoals documenten, dagboekinzendingen, afspraken ... toevoegen. Al deze informatie moet ergens opgeslagen worden en vanuit die opslagplaats weer geraadpleegd kunnen worden, en dit vanaf verschillende apparaten. Een database is hiervoor de beste optie. In deze toepassing hebben we een \textit{real-time} database\footnote{Een real-time database is een database waarin aanpassingen quasi onmiddellijk worden doorgevoerd, zodat deze aanpassingen ook op andere apparaten snel zichtbaar zijn.} nodig, zodat onder andere chatberichten zo snel mogelijk gesynchroniseerd kunnen worden.
	\\\\
	Firebase leek ons de beste optie. Dit is een platform dat verschillende producten huisvest om het creëren van apps en websites te vergemakkelijken. De Realtime database en de Cloud Firestore zijn de twee real-time databases uit deze producten. Daarnaast zullen we ook het beveiligd authenticatiesysteem en de opslag van foto's en video's in de cloud gebruiken. 
	\\
	Firebase heeft als voordeel dat het een populair platform is, waardoor er ook een uitgebreide online documentatie \cite{Firebase} beschikbaar is. Het platform is echter maar beperkt gratis. Wegens het verwachte lage gebruikersaantal van de app en de website zal er echter geen of slechts een zeer beperkte kostprijs zijn.
	\\
	We verkiezen uiteindelijk Cloud Firestore als database. Over het algemeen \cite{Firebase} wordt deze beschouwd als een verbetering van de oudere Realtime database. Daarnaast is het filteren van data in Firestore gemakkelijker.	
	\newpage
	\section{Resultaten}
	%Na onze studie van de verschillende tools, waren we eraan uit welke we gingen gebruiken: Android Studio voor de app, \verb!HTML!, \verb!CSS! en \verb!JavaScript! in combinatie met \verb!Node.js! voor de website en Cloud Firestore als database.
	%\\\\
	%Daarna hebben we gezocht naar de structuur (die weergegeven wordt op \pageref{flowchart}), die we wilden bereiken voor het communicatieplatform. 
	Voor het communicatieplatform hebben we
	%Hiervoor hebben we 
	verschillende pagina's nodig: login, menu, chatoverzicht, documenten, kalender, dagboek, activiteiten toevoegen, overzicht van activiteiten, documenten toevoegen en chatberichten. Het pleeggezin kan chatten, documenten lezen, hun kalender bekijken, een afspraak toevoegen en het dagboekje bekijken. Voor het dagboekje zijn er meerdere opties, ofwel wordt er een activiteit toegevoegd, ofwel wordt er een overzicht gevraagd van een bepaalde activiteit. Daarnaast moet de trainer ook bepaalde functies kunnen uitvoeren. Hij kan ook chatten en het dagboek bekijken voor een specifieke hond. Daarnaast kan hij zelf ook documenten uploaden en afspraken toevoegen in de kalender. Hiervoor kiest hij of hij dat voor een specifieke puppy wil doen of voor alle puppy's tegelijk. Hiervoor vonden we de structuur (weergegeven op \pageref{flowchart}).
	\subsection{Database}
	Zowel de app als website zijn verbonden met de database Cloud Firestore. Hier worden alle gegevens die worden geüpload of ingegeven via de applicatie en website, gestructureerd opgeslagen . Het is dus een plek waar alle data te vinden is, maar ook een $ontmoetingsplaats$ van de website en app. Alle gegevens van beide tools komen hier samen. Op die manier kunnen alle data ook opgevraagd worden via website en app. \\
	Bij zo'n database is vooral de structuur belangrijk. Op pagina \pageref{programmeerstructuur} in de appendices is een overzicht te zien van hoe we de gevarieerde informatie gaan wegschrijven vanuit de app en website naar de database. Hierbij maken we verschillende collecties aan: Dagboekje, Berichten, Kalender, Documenten en Gebruikers. Elke collectie bevat verschillende items met elk hun eigen parameters$/$gegevens. 
	Afbeeldingen en pdf-documenten worden eerst opgeslagen in de Cloud Storage van Firebase. Daarna wordt er een link naar de opslagplaats in de database geplaatst. Bij pdf-documenten wordt ook de naam en de grootte van het document opgeslagen, aangezien we deze zullen gebruiken in de lay-out. Aan de hand van de link kunnen afbeeldingen uit de Cloud Storage weergegeven worden in de app en de website en kunnen pdf-documenten gedownload worden.  
	
	\subsection{Login-pagina}
	Via de login-pagina kan je met een gekregen account via e-mailadres en wachtwoord inloggen. Dit account staat gelinkt met je pleeghond. Op die manier komt elke gebruiker bij de data terecht dat over hun hond gaat. %Voor de trainer gaat het net een beetje anders. Hij zal bij sommige functies kunnen kiezen voor welke hond hij informatie wil opvragen, maar daar komen we later op terug. \\
	%Hij kan van elke hond de data opvragen, hiervoor moet hij telkens aangeven voor welke hond hij verder wil gaan. Wil hij de gegevens van één specifieke hond dan selecteert hij die naam. Indien hij meerdere honden tegelijk wil opvragen, kan hij kiezen voor alle honden.
	%Indien het inloggen succesvol gelukt is, kom je op de hoofdpagina terecht. Daar zijn de vier functies van de applicatie en website terug te vinden: chat, documenten, kalender en dagboek. 
	Voor het inloggen wordt gebruik gemaakt van \textit{Firebase Authentication}, een ingebouwde functionaliteit van Firebase. Zo worden de wachtwoorden veilig bewaard. Elke gebruiker kan met een door ons gekregen account via een email en bijhorend wachtwoord inloggen, zoals weergegeven in figuur \ref{App_login} voor de app en in figuur \ref{website_login} voor de website. Wanneer een juiste email-wachtwoord combinatie wordt ingegeven, wordt de gebruiker doorverwezen naar de hoofdpagina (figuur \ref{App_homepage}) bij de app of naar de chat bij de website. \\Op de hoofdpagina van de applicatie zijn de vier hoofdfuncties terug te vinden: chat, documenten, kalender en dagboek. Elke pagina van de website bevat een navigatiebalk zodat er telkens eenvoudig kan gewisseld worden tussen de functionaliteiten en een hoofdpagina overbodig wordt.
	\begin{figure}[h]
		\begin{minipage}[t]{0.4\linewidth}
			\includegraphics[width=3.5cm]{App_login}
			\caption{Login-pagina}\label{App_login}
		\end{minipage}\hfill
		\begin{minipage}[t]{0.4\linewidth}
			\includegraphics[width=3.5cm]{App_homepage}
			\caption{Hoofdpagina}\label{App_homepage}
		\end{minipage}
	\end{figure}
	\begin{figure}[h]
	\centering
	\includegraphics[width=0.735\textwidth]{screen_web_login}
	\caption{Om in te loggen worden het email adres en het wachtwoord gevraagd.}\label{website_login}
	\end{figure}
	
	\subsection{Chat}
	Indien er doorgeklikt wordt op chat, wordt er een overzicht (figuur \ref{App_chatoverzicht} voor de app en figuur \ref{web_chat} voor de website) getoond van de alle gebruikers. Via de chat is er de optie om berichten (en foto's) te versturen naar andere gebruikers. Bovenaan is er een zoekbalk (zie figuur \ref{App_zoekfunctie}) te vinden, die ervoor zorgt dat chats makkelijk teruggevonden kunnen worden indien het overzicht van gebruikers lang is. Hierbij kan er doorgeklikt worden naar het gesprek met die gebruiker. Elke conversatie wordt uniek opgeslagen in de database door een combinatie van de twee gebruikers-id's\footnote{Elke gebruiker wordt uniek bepaald door een unieke string, deze string wordt de gebruikers-id genoemd}. Voor elk bericht wordt er opgeslagen wie de verzender en ontvanger is, berichtinhoud, datum$/$tijd en eventueel een referentie naar de media. Dankzij de kleur van de berichten (zoals weergegeven op figuur \ref{App_chat}), wordt ook een onderscheid gemaakt tussen verzonden en ontvangen berichten. \\\\
	%Voor elke conversatie wordt ook het laatst verzonden bericht opgeslagen. Deze wordt telkens overschreven wanneer er een nieuw bericht wordt verzonden. Voor dit bericht wordt er opgeslagen of dit bericht gelezen is ofniet. Een bericht wordt als gelezen beschouwd wanneer de conversatie met de gebruiker wordt geopend. Bij de website  wordt elk laatst verzonden bericht wordt weergegeven onder de naam van de geselecteerde gebruiker. \\
	Zo kan er snel gecommuniceerd worden tussen het pleeggezin en de trainer mochten er bijkomende problemen of vragen zijn, omtrent bijvoorbeeld de volgende training. Trainers kunnen ook onderling met elkaar chatten, zodat ze makkelijk kunnen afspreken voor bijvoorbeeld een meeting. Indien pleeggezinnen willen afspreken om bijvoorbeeld samen te gaan wandelen, kunnen ze ook elkaar bereiken.
	\begin{figure}[h]
		\begin{minipage}[t]{0.33\linewidth}
			\includegraphics[width=3.5cm]{App_chat2}
			\caption{Chatoverzicht}\label{App_chatoverzicht}
		\end{minipage}\hfill
		\begin{minipage}[t]{0.33\linewidth}
			\includegraphics[width=3.5cm]{App_chat}
			\caption{Zoekfunctie}\label{App_zoekfunctie}
		\end{minipage}\hfill
		\begin{minipage}[t]{0.3\linewidth}
			\includegraphics[width=3.5cm]{App_chat3}
			\caption{Chat}\label{App_chat}
		\end{minipage}
	\end{figure}
	\begin{figure}[h]
	\centering
	\includegraphics[width=0.735\textwidth]{screen_web_chat.jpg}
	\caption{Er wordt links een overzicht gegeven van chats met de andere gebruikers van het platform: weergegeven door hun naam, (1) aanduiding voor een ongelezen bericht, (2) weergave van het laatste bericht (ontvangen), of (3) laatste bericht (verstuurd) en (6) de zoekbalk.
	Rechts is de weergave van de chat met (4) een verzonden bericht en (5) een ontvangen bericht.}
	\label{web_chat}
	\end{figure}
	
	\subsection{Documenten}
	Bij documenten krijgt een lid van het pleeggezin een chronologisch overzicht (zie figuur \ref{App_document} voor de app en figuur \ref{web_documenten} voor de website) van verschillende documenten. Zij zijn opgeslagen in de database en kunnen gedownload worden. Die documenten zijn specifiek gericht voor het gezin van een bepaalde pleeghond. Geen enkel ander gezin kan aan deze informatie. De documentnaam en -grootte zijn hierbij weergegeven. Er is ook een knop voor het downloaden. \\
	De trainer heeft diezelfde documenten geüpload. Hij kiest voor wie hij bestanden wil uploaden (voor de app is dit weergegeven op figuur \ref{App_trainerdoc}): voegt hij documenten toe die enkel door een bepaalde gebruiker kunnen worden gezien of algemene documenten die door elke gebruiker kunnen worden gezien? Daarna moet de trainer een pdf-bestand selecteren, wat duidelijk wordt weergegeven op de app \ref{App_trainerdoctoev}. Het is dus enkel mogelijk om pdf-bestanden te uploaden. Dan moet hij deze uploaden, op dat moment worden die documenten ook geüpload naar de database. \\
	De trainer kan ook een overzicht van de documenten opvragen voor een specifieke hond, bij de app staat daarbij rechtsonder ook nog een knop om rechtstreeks een document up te loaden \ref{App_trainerdocoverz}. Elk document wordt gekenmerkt door een datum waarop het toegevoegd werd, een referentie naar de Firebase Storage, een naam en grootte. En wordt opgeslagen onder volgens het unieke gebruikers-id voor de lid van het pleeggezin waarvoor het bedoeld is. De trainer kan per pleeggezin de documenten opvragen en hierbij ook deze bestanden downloaden. 
	Op de website kan er ook nog gezocht worden via een zoekbalk naar een bepaald document op basis van de documentnaam. \\
	\begin{figure}[h]
		\begin{minipage}[t]{0.2\linewidth}
			\includegraphics[width=3cm]{App_documenten}
			\caption{Documenten- overzicht}\label{App_document}
		\end{minipage}\hfill
		\begin{minipage}[t]{0.2\linewidth}
			\includegraphics[width=3cm]{App_documenten3}
			\caption{Documenten menu trainer}\label{App_trainerdoc}
		\end{minipage}\hfill
		\begin{minipage}[t]{0.2\linewidth}
			\includegraphics[width=3cm]{App_documenten2}
			\caption{Documenten- overzicht \\trainer}\label{App_trainerdocoverz}
		\end{minipage}\hfill
		\begin{minipage}[t]{0.2\linewidth}
			\includegraphics[width=3cm]{App_documenten4}
			\caption{Documenten toevoegen trainer}\label{App_trainerdoctoev}
		\end{minipage}
	\end{figure}
	\begin{figure}[h]
		\centering
		\includegraphics[width=0.735\textwidth]{screen_web_documenten.jpg}
		\caption{Bij het documentenoverzicht hebben we (1) de optie om documenten van pleeggezinnen te bekijke (enkel voor trainers), (2) de zoekfunctie, (3) de optie om documenten toe te voegen (enkel voor trainers) en (4) het overzicht van de documenten, met de optie om het document te downloaden.}\label{web_documenten}
	\end{figure}

	\subsection{Kalender}
	Via de hoofdpagina kan er ook gekozen worden voor kalender. Op dat moment wordt je doorverwezen naar een pagina waar je een chronologisch overzicht krijgt van de verschillende activiteiten staan gepland. Hierbij is er een duidelijk onderscheid te maken tussen de app en website. \\
	Via de applicatie krijg je een overzicht van de toekomstige activiteiten (zie figuur \ref{App_kalender}) in de vorm van een chronologische lijst met alle activiteiten. Hierbij zijn een omschrijving van de afspraak, de datum, het begin- en einduur, de locatie en vrije opmerkingen onmiddellijk leesbaar. Het is ook mogelijk om een nieuwe afspraak te maken \ref{App_afsprtoev}. Hiervoor moet je een omschrijving, datum, beginuur en einduur kiezen. Indien gewenst kan ook een locatie of extra opmerkingen toegevoegd worden. Voor de datum kan je via een kalender de geplande dag kiezen en het begin- en einduur kunnen gekozen worden via een klok. Als je hiermee klaar bent, wordt de data terug doorgestuurd naar de database. Vanaf dat moment zijn ze toegevoegd aan het overzicht van activiteiten op zowel de app als website. \\
	Analoog aan de app krijg je een overzicht van alle activiteiten (zie figuur \ref{web_kalender}), maar deze keer in een $'$echte$'$ kalendervorm.
	
	
	In de kalendervorm zijn de omschrijving van de afspraak, het begin- en eindmoment visueel zichtbaar. Wanneer er dubbel geklikt wordt op een afspraak, worden de locatie en de vrije opmerkingen zichtbaar en ook een optie om de afspraak te verwijderen. Door een afspraak te verwijderen worden al zijn gegevens uit de database gewist. Deze wijziging is meteen zichtbaar op het overzicht in zowel de applicatie als website.
	\\
	Indien een trainer op het kalenderoverzicht wil terechtkomen, moet hij kiezen opnieuw eerst kiezen voor wie deze informatie wil (Dit is voor de app weergegeven in figuur \ref{App_trainerkal}). Hij heeft hierbij twee opties: zichzelf of een specifieke hond. Indien je een lid bent van het pleeggezin krijg je onmiddellijk de afspraken voor jullie pleeghond. 
	\begin{figure}[h]
		\begin{minipage}[t]{0.33\linewidth}
			\includegraphics[width=2.5cm]{App_kalender2}
			\caption{Kalender}\label{App_kalender}
		\end{minipage}\hfill
		\begin{minipage}[t]{0.33\linewidth}
			\includegraphics[width=2.5cm]{App_kalender}
			\caption{Afspraak \\toevoegen}\label{App_afsprtoev}
		\end{minipage}\hfill
		\begin{minipage}[t]{0.3\linewidth}
			\includegraphics[width=2.5cm]{App_kalender3}
			\caption{Kalender menu trainer}\label{App_trainerkal}
		\end{minipage}
	\end{figure}
	\begin{figure}[h]
	\centering
	\includegraphics[width=0.735\textwidth]{screen_web_kalender.jpg}
	\caption{De activiteiten worden weergegeven in een kalendervorm (4).Na dubbelklikken wordt extra informatie zichtbaar en is het mogelijk de afspraak te verwijderen (3). Er kunnen ook afspraken worden toegevoegd (1). Trainers hebben de optie (2) om kalenders van pleeggezinnen te bekijken.}\label{web_kalender}
	\end{figure}
	
	\newpage
	\subsection{Dagboek}
	Het dagboek is ook te vinden via de hoofdpagina. 
	Op de app kunnen we ervoor kiezen om een overzicht te krijgen van oefeningen of om een oefening toe te voegen \ref{App_dagboek}. Indien we voor het eerste kiezen, kunnen we via een zoekfunctie een activiteit selecteren waarvoor we een overzicht willen verkrijgen. Indien er geen enkele activiteit wordt geselecteerd, worden alle activiteiten weergeven in het overzicht zoals op figuur \ref{App_dagboekoverzicht}. Hierbij zijn de volgende gegevens zichtbaar: naam, datum, alle stresssignalen\footnote{Het is de bedoeling dat de hond leert om zo weinig mogelijk stress te hebben, aan de hand van stresssignalen zoals bijvoorbeeld rillen, kan er ingeschat worden hoe de hond zich voelt.}, stressniveau en een foto van de oefening. Indien we kiezen om een activiteit van diezelfde dag toe te voegen zoals op figuur \ref{App_dagboekacttoev}, kunnen we dit doen door verder te klikken. Hier kunnen de naam, één of meerdere stresssignalen, het stressniveau en een omschrijving worden meegegeven. Er kan ook een foto worden geüpload. Eenmaal dit allemaal is gebeurd, is de oefening terug te vinden in het overzicht tussen alle andere oefeningen. \\
	Op de website kunnen we meteen een activiteit toevoegen, zie figuur \ref{web_dagboektoev}. Hierbij kunnen er opnieuw dezelfde gegevens ingegeven worden. Door verder te klikken kunnen we het overzicht van de oefeningen bekijken. Indien we slechts een overzicht willen van één specifieke oefening, kunnen we deze selecteren. Op de website is er nog een extra tool te vinden, namelijk het gemiddelde stressniveau per oefening. Daar zie je een overzicht van alle gemiddelden van de verschillende soorten oefeningen. Zo kan er makkelijk worden gekeken naar welke er nog meer moeten getraind worden en welke er al helemaal goed lukken. \\
	Indien je een trainer bent, kan je opnieuw kiezen voor wie je het dagboekje wil zien (weergegeven op de app zoals \ref{App_trainerdagb}). Dit kan handig zijn voor het pleeggezin van die ene hond. Want kan de trainer zo beter volgen in het traject van de opleiding en kan indien nodig wat tips geven aan het pleeggezin. De trainer kan zelf geen activiteiten toevoegen, aangezien trainingen niet tot het dagboek behoren en hij dus geen dagelijkse oefeningen uitvoert met de hond. Daarom is er op figuur \ref{App_trainerdagbmenu} van de app geen mogelijkheid om een activiteit toe te voegen.\\
	\begin{figure}[h]
		\begin{minipage}[t]{0.17\linewidth}
			\includegraphics[width=2.4cm]{App_dagboek}
			\caption{Dagboek menu}\label{App_dagboek}
		\end{minipage}\hfill
		\begin{minipage}[t]{0.17\linewidth}
			\includegraphics[width=2.4cm]{App_dagboek3}
			\caption{Dagboek}\label{App_dagboekoverzicht}
		\end{minipage}\hfill
		\begin{minipage}[t]{0.17\linewidth}
			\includegraphics[width=2.4cm]{App_dagboek2}
			\caption{Activiteit\\ toevoegen}\label{App_dagboekacttoev}
		\end{minipage}\hfill
		\begin{minipage}[t]{0.17\linewidth}
			\includegraphics[width=2.4cm]{App_dagboek5}
			\caption{Dagboek trainer}\label{App_trainerdagb}
		\end{minipage}\hfill
		\begin{minipage}[t]{0.17\linewidth}
			\includegraphics[width=2.4cm]{App_dagboek4}
			\caption{Trainer menu}\label{App_trainerdagbmenu}
		\end{minipage}
	\end{figure}
	\begin{figure}[h]
		\begin{minipage}[t]{0.4\linewidth}
			\centering
			\includegraphics[width=0.7\textwidth]{screen_web_dagboek2.0.jpg}
			\caption{Om een oefening toe te voegen moeten verschillende gegevens worden ingegeven. Er kan ook worden doorgeklikt naar het overzicht.}
			\label{web_dagboek2}
		\end{minipage} \hfill
		\begin{minipage}[t]{0.6\linewidth}
			\includegraphics[width=1\textwidth]{screen_web_dagboek.jpg}
			\caption{\\Hier wordt het dagboek weergegeven. Daarbij is er (1) een filteroptie, (2) statistieken van de stressniveaus en (3) weergave van de oefeningen.}
			\label{web_dagboek}
		\end{minipage}
	\end{figure}


	%\subsection{App}
	%De app is verbonden met Cloud Firestore. 
	%Via de login-pagina kun je met een gekregen account via e-mailadres en wachtwoord inloggen. Dit account staat gelinkt met je pleeghond. Op de hoofdpagina zijn de vier grote functies terug te vinden: chat, documenten, kalender en dagboek. 
	%Indien er doorgeklikt wordt op chat, wordt er een overzicht getoond van de voorbije chats. Bovenaan is er een zoekbalk te vinden die ervoor zorgt dat chats makkelijk worden teruggevonden indien het overzicht van chats lang is. Hierbij kan er doorgeklikt worden naar het gesprek met een andere gebruiker. Die berichten worden upgeload naar en vanuit de database. 
	%Bij documenten krijgt een lid van het pleeggezin een overzicht van de verschillende documenten die opgeslagen zijn in de database en die te downloaden zijn. De trainer heeft die documenten geupload. Hij moet eerst een file selecteren, wat duidelijk wordt weergegeven op de app. Daarna moet hij deze uploaden naar de database.
	%Er kan ook gekozen worden om naar de kalender te kijken.
	%Het dagboek is ook te vinden via de hoofdpagina.
	
	%\subsection{Website}
	%Met diezelfde database is het ook gelukt om de website te verbinden. Het login-systeem werkt op een gelijke manier als de app. Je kan inloggen met een e-mailadres en wachtwoord die geconnecteerd staan met je hond indien je een lid bent van het pleeggezin. Als je een trainer bent, kun je bij elke specifieke actie een hond kiezen waarvoor je dit wilt uitvoeren.
	
	%\subsection{Database}
	%Cloud Firestore is dus gekoppeld met zowel de app als de website. Bij zo'n database is vooral de structuur belangrijk. Op \ref{programmeerstructuur} is een overzicht te zien van hoe we de gevarieerde informatie gaan wegschrijven vanuit de app en website naar de database. Hierbij maken we verschillende collecties aan: Dagboekje, Berichten, Kalender, Documenten en Gebruikers. Elke collectie bevat verschillende items met elk hun eigen parameters/gegevens. 
	\newpage
	\label{flowchart}\includepdf[pages=-]{Flowchart}
	
	\newpage
	\section{Ontwikkelingen in de toekomst}
	\subsection{Verbeteringen}
	Voor de applicatie zijn er nog verschillende verbeteringen mogelijk. 
	De functies voor trainer kunnen nog worden geoptimaliseerd. Zo kan ervoor gezorgd worden dat een trainer niet op verschillende pagina's telkens moet doorklikken, maar dit allemaal op eenzelfde pagina kan. Daarnaast is het onderscheid tussen trainer en pleeggezin momenteel niet efficiënt geïmplementeerd. Een betere methode zou de opslaggrootte van de app sterk kunnen verminderen.
	Ten tweede kunnen in de chat van de app zowel berichten als foto's worden ontvangen, maar tot nu toe kunnen enkel berichten worden verstuurd. Het versturen van foto's kan wel gedaan worden op de website.\\

	Voor de website zijn er minder gebreken. Hier kan vooral het design nog geoptimaliseerd worden, momenteel is de website niet compatibel op mobiele apparaten. \\
	
	Daarnaast zouden ook de mogelijke kosten voor de database verminderd kunnen worden. Zo kunnen er mechanismen ingebouwd worden om het aantal keer dat er data uit de Cloud Firestore ingelezen wordt, te beperken. Dit is immers een van de zaken die de kosten bepaald. Ook een overstap naar de Realtime Database zou een oplossing kunnen bieden. Dit kan echter andere nadelen tot gevolg hebben, zie[bron]. 

	\subsection{Extra functionaliteiten}
	Er zijn nog verschillende ontwikkelingen mogelijk die we niet hebben uitgevoerd omwille van tijdsgebrek.
	Ten eerste zijn er enkele vereisten die we niet hebben kunnen bereiken. Zo kunnen foto's in de chat van de app enkel ontvangen worden, maar niet verstuurd. Dit kan wel gedaan worden op de website. Video's kunnen momenteel op geen van beide platformen verstuurd worden in de chat, en ze kunnen ook niet in het dagboekje worden geplaatst. Tot slot kan er in het dagboek van de app niet gefilterd worden op stressniveau.
	Ten tweede kunnen er nog functionaliteiten worden ingebouwd die het gebruiksgemak zouden verbeteren. 
	Meldingen vanuit de app voor kalenderafspraken, nieuw chatberichten of beschikbare documenten zou het leven van de gebruikers vergemakkelijken. Zo hoeven de pleeggezinnen media niet constant te checken om toch op de hoogte te zijn van de nieuwste wijzigingen.
	Ook de mogelijkheid om specifieke afspraken, documenten en activeiten te verwijderen zou geïmplementeerd kunnen worden. Momenteel kan men enkel afspraken verwijderen en dit enkel via de website.
	Het zou ook handig zijn als trainers en pleeggezinnen vriendenlijst konden aanmaken. Op deze manier zullen ze geen overbodige spam in de chat hebben van mensen waarmee ze niet bevriend zijn.
	Daarnaast kan de trainer op dit moment enkel een document uploaden voor ofwel iedereen ofwel één puppy. In de toekomst zou het mogelijk zijn om documenten toe te voegen voor bijvoorbeeld twee puppy's in één keer. Dit zou de trainer veel tijd besparen.
	Tot slot hebben pleeggezinnen met meerdere honden op dit moment verschillende accounts nodig. Het zou voor hen makkelijker zijn om dit te kunnen combineren via één account. Dit zou kunnen door bij dagboek, documenten en kalender een onderscheid te maken tussen de honden. En dus te vragen voor welke hond het gezin de informatie wil. Dit zou gelijkaardig zijn aan de trainer-functie. Hierbij wordt er ook gevraagd voor welke hond er data wil worden geüpload of opgevraagd.
	
	\newpage
	\section*{Conclusie}
	Het doel is om een communicatieplatform te vormen voor trainers en pleeggezinnen van blindengeleidehonden in wording. Honden in opleiding logeren bij pleeggezinnen en worden doorheen het dagelijks leven getraind. Tijdens die training is er niet altijd een optimale communicatie met de trainer. Daarvoor ontwikkelen wij een app en website die  via een database in connectie staan met elkaar.
	\\\\
	Om dit te realiseren moesten we eerst de tools zoeken die we konden gebruiken voor de app en website. Dit nam veel tijd in beslag aangezien de keuze een grote invloed kan hebben op het resultaat. De tools moeten kunnen voldoen aan alle eisen van de probleemstelling. Daarna was het belangrijk om kennis te maken met de werking van de tools. Voordat we begonnen aan het ontwerpen van de app en website, bouwden we een kennis op van het gebruik van de programma's. Toen we dit onder de knie hadden, konden we beginnen met het echte werk. We brainstormden over hoe we het project het best konden aanpakken. We dachten na over de structuur van de app en website en bedachten ook hoe we de verschillende functies die de trainers en pleeggezinnen moeten kunnen doen, gingen implementeren. Tot slot begonnen we met het design en het implementeren van de app en website.
	\\\\
	Er is nog ruimte voor optimalisatie bij de werking van de applicatie en bij het design van de website. Aangezien we in tijdsnood kwamen hebben we gekozen voor een optimale werking van de website en een optimaal design app, maar vooral voor een compatibele werking van beide tools met de database.\\\\
	%$Gelukkig bleven we ons best doen en vond Charlotte ons een topteam! :) $
	
	
	\newpage
	\begin{appendices}
		\section{Planning} \label{planning}
		 De app en de website bestonden uit dezelfde vijf hoofdfuncties van onze communicatieplatform: inlogsysteem, dagboek, kalender, documenten en chat. Voor elk van deze onderdelen moesten we ons zowel op het design als op de implementatie focussen. Daarbovenop moest het onderscheid tussen trainer en pleeggezin worden gecreëerd. Ook in de database moesten diezelfde vijf functies worden geïmplementeerd. Tot slot moest alles getest en afgestemd worden op elkaar. \\
		De rapportering van dit project konden we onderverdelen in drie stukken: tussentijds verslag, eindverslag en presentatie. Bij deze eerste twee verantwoordden we de tools die we gebruikten, verklaarden we uitgebreid de resultaten en toonden we eventuele verbeteringen$/$ontwikkelingen voor de toekomst.\\
		Voor de presentatie was het grootste werk kijken hoe we onze media efficiënt zouden kunnen tonen. We hebben ook goed besproken hoe we de demo zouden aanpakken.
		\\\\
		Onze oorspronkelijke planning is te vinden op pagina \pageref{gantt-grafiek} in deze appendices. In de loop van het project zijn we afgeweken van deze planning, aangezien we doorhadden dat dit niet de beste volgorde was om het project uit te voeren.\\
		
		 We zijn begonnen met de tools te zoeken en daarna de pro's en contra's kort samen te vatten, zodat we deze later in een degelijk verslag konden gieten. Daarna zijn we begonnen met zowel de app als de website. De volgorde van het creëren van beide tools zijn wel gebeurd zoals aangegeven in de planning. Hoewel het design van de app nog vele keren moest worden aangepast, was deze voor het grote deel al klaar voordat we aan het tussentijds verslag begonnen. Tijdens het implementeren van de app en website werd ook de database geprogrammeerd. En op het einde van ons project werden de database-collecties van de applicatie en website op elkaar afgestemd en getest.\\ 
		Ondertussen werd ons eindverslag gemaakt, het intensief werken aan de verdere uitwerking werd wel veel later gestart dan voorzien. We zijn wel op het vooraf-vastgelegde moment gestart met het maken van de presentatie.
		
		\section{Samenwerking}
		\subsection{Verantwoordelijkheden}
		Bij het begin van het project werden er rollen en dus verantwoordelijkheden gecreëerd. Thomas nam de verantwoordelijkheid voor het ontwikkelen van de website. Daarnaast heeft Ruben de verantwoordelijkheid van de app op zich genomen. Tot slot is Manon verantwoordelijk voor het design van het platform en voor de rapportering van ons project.
		\subsection{Taakverdeling}
		Wij hebben samen gezocht naar de verschillende tools die nodig waren om een website en app te bouwen en met elkaar te verbinden via een database. Nadat we deze hadden gevonden, werd de Gantt-grafiek (in bijlage op pagina \pageref{gantt-grafiek}) opgesteld en hebben we de taken (op pagina \pageref{planning}) opgesplitst. 
		\\\\
		Thomas heeft eerst gezorgd voor de connectie tussen de website en database. Daarna is hij begonnen met het implementeren van de website. 
		Ruben heeft gekeken voor het verbinden van de applicatie en database. Hij heeft er ook voor gezorgd dat Android Studio via GitHub functioneert om zo een vlotter groepswerk te creëren. Daarna is hij begonnen met het implementeren van de app. 
		Manon heeft zich eerst bezig gehouden met de app te ontwerpen, meer bepaald het design van de verschillende pagina's en de doorverwijzingen tussen de pagina's. Daarna is ze begonnen met het tussentijds verslag te schrijven. \\
		Voor het maken van het tussentijds verslag hebben we later ook de taken verdeeld. We namen elk een of meerdere onderdelen voor zich. Iedereen kreeg het onderdeel waarvan hij of zij het meest vanaf wist. Tot slot heeft iedereen het verslag doorgelezen en goedgekeurd.
		\\\\
		Toen dat tussentijds verslag ingediend was, hebben Thomas en Ruben gekeken om de database-collecties van de app en website op elkaar af te stemmen. Ondertussen heeft Manon verder de applicatie geïmplementeerd.
		Tot slot heeft Ruben verdergewerkt aan de app, terwijl Thomas en Manon bezig waren met het eindverslag en de presentatie voor te bereiden. \\
		Op het einde van ons project heeft iedereen het verslag doorgelezen en goedgekeurd.
		
		\section{Gantt-grafiek}
		De gantt-grafiek is te vinden op pagina \pageref{gantt-grafiek} in deze appendices.
		\section{Structuur database}
		De structuur die we gebruikt hebben in onze database en dus tijdens het programmeren is te vinden op pagina \pageref{programmeerstructuur} in deze appendices.
		
		\newpage
		\label{gantt-grafiek}\includepdf[pages=-]{Ganttchart}
		\label{programmeerstructuur}\includepdf[pages=-]{Eind_Diagram_1.drawio}
		\includepdf[pages=-]{Eind_Diagram_2.drawio}
	\end{appendices}
	
	\bibliography{Bibliografie}
	\bibliographystyle{unsrt}
	\nocite{*}
\end{document}